\documentclass[noamssymb,svgnames]{beamer}
\usecolortheme{beaver}
\useinnertheme[shadow]{rounded}
\usefonttheme{serif}
\usefonttheme{professionalfonts}

\usepackage[bitstream-charter]{mathdesign} % Use BT Charter font
\usepackage{beramono}                      % Use Vera Mono for \ttfamily
\usepackage[T1]{fontenc}                   % Use T1 encoding instead of OT1
\usepackage[utf8]{inputenc}                % Use UTF8 input encoding
\usepackage{microtype}                     % Improve typography
\usepackage{booktabs}
\usepackage[binary-units]{siunitx}
\usepackage{tikz}
\usetikzlibrary{shapes.geometric}

\usepackage{hyperref}
\hypersetup{pdfstartview=Fit}

% BEAMER CONFIGURATION ---------------------------------------------------------
\setbeamerfont{block title}{size=\normalsize}
\setbeamerfont{block body}{size=\scriptsize}
\setbeamercolor{block title}{fg=darkred,bg=gray!10!white}
\setbeamercolor*{item}{fg=darkred}

% TCOLORBOX / MINTED CONFIGURATION ---------------------------------------------
\usepackage[many,minted]{tcolorbox}
\NewTCBListing{python}{ O{} }{listing only,minted language=python,#1}
\tcbset{listing only}
\setminted{
  mathescape,
  autogobble,
  fontfamily=courier,
  framesep=2mm
}

\definecolor{darkblue}{rgb}{0.0,0.0,0.6}
\newcommand{\obj}[1]{\texttt{\color{darkblue}#1}}

% -----------------------------------------------------------------------------
\title{Python 101}

\institute{\includegraphics[width=2in]{../images/openmc_logo.png}}

\date{OpenMC Workshop \\ Canadian Nuclear Laboratories \\ March 14--16, 2017 }

% -----------------------------------------------------------------------------
\begin{document}

\frame{\titlepage}

% -----------------------------------------------------------------------------
\begin{frame}{Python}
  Python is often referred to as a ``scripting'' language
  \begin{itemize}
  \item Dynamically typed (type of variable interpreted at runtime)
  \item Implicitly typed (type of variable never declared)
  \item Case-sensitive (var and VAR and different)
  \item Object oriented (everything is an object)
  \item Containers are zero-indexed
  \item Whitespace used to denote blocks (rather than \{ \} as in other
    languages)
  \item Traditional implementation (CPython) is interpreted (it doesn't have to
    be)
  \end{itemize}
  \vfill
  \begin{tcolorbox}[title=NOTE]
    This introduction is based on Python 3.
  \end{tcolorbox}
\end{frame}

\begin{frame}{Built-in types}
  \begin{itemize}
  \item Boolean: \obj{bool}
  \item Numeric types: \obj{int}, \obj{float}, \obj{complex}
  \item Text sequence: \obj{str}
  \item Sequence types: \obj{list}, \obj{tuple}
  \item Set types: \obj{set}
  \item Mapping types: \obj{dict}
  \end{itemize}
  % Sets and dicts are not ordered
\end{frame}

\begin{frame}{Numeric types}
  \begin{itemize}
  \item Operations: +, -, *, /, //, \%, **
  \item Functions: \obj{abs}, \obj{math.sqrt}, \obj{math.exp},
    \obj{math.cos}, ...
  \end{itemize}
  % Show basic use of operators
  % Show conversions between types
  % Show += type operators
\end{frame}

\begin{frame}{Strings}
  \begin{itemize}
  \item \obj{'...'}, \obj{"..."}, \obj{"""..."""}
  \item Methods: \obj{upper()}, \obj{find()}, \obj{strip()},
    \obj{split()}, ...
  \item Get length with \obj{len()}
  \item Can add, multiply strings
  \item \obj{format()} method can be used to substitute variables in strings
  \end{itemize}
\end{frame}

\begin{frame}{Booleans}
  \begin{itemize}
  \item \obj{True} and \obj{False}
  \item Result from comparison operators: <, <=, >, <=, !=, ==, \obj{is},
    \obj{is not},
  \item Used in \obj{if} and \obj{while} loops
  \item Logical operators: \obj{and}, \obj{or}, \obj{not}
  \end{itemize}
\end{frame}

\begin{frame}{Sequence types}
  \begin{itemize}
  \item List: an ordered, mutable collection of objects
    \begin{itemize}
    \item Methods: \obj{append()}, \obj{insert()}, \obj{reverse()},
      \obj{pop()}, \obj{sort()}, ...
    \item Construct with \obj{list()} or \obj{[...]}
    \end{itemize}
  \item Tuple: an ordered, immutable collection of immutable objects
    \begin{itemize}
    \item Construct with \obj{tuple()} or \obj{(...)}
    \end{itemize}
  \item Can get length with \obj{len()}
  \item Indexed from 0
  \item Negative indexing can be used to count from end
  \item Can refer to ``slices''
  \end{itemize}
  % Can add lists together
\end{frame}

\begin{frame}[fragile]{Sets}
  \begin{itemize}
  \item Unordered collection of distinct (hashable) objects
  \item Construct with \obj{set()} or \obj{\{'s', 3, 2.0\}}
    \begin{itemize}
    \item However, \obj{\{\}} is not a set
    \end{itemize}
  \item Methods: \obj{add()}, \obj{difference()}, \obj{intersection()},
    \obj{union()}
  \item Can also use operators: \obj{in}, \obj{\&}, \obj{|}, \obj{-}, \obj{\^}
  \item Get length with \obj{len()}
  \end{itemize}
\end{frame}

\begin{frame}[fragile]{Dictionaries}
  \begin{itemize}
  \item Mapping or associative array
  \item Indexed by \emph{key} rather than by a range of numbers
  \item Construct with \obj{dict()} or \obj{\{'a': 1, 'b': 2, ...\}}
  \item Set item:
    \begin{python}
      person = {'eyecolor': 'blue', 'age': 15}
      person['name'] = 'John'
      print(person['age'])
    \end{python}
  \item Iterate with \obj{keys()}, \obj{values()}, \obj{items()}
  \item Other methods: \obj{get()}
  \item Delete a key with \obj{del}
  \item Get length with \obj{len()}
  \item \obj{in} operator compares against keys, not values
  \end{itemize}
\end{frame}

\begin{frame}[fragile]{Conditional statement}
  \begin{itemize}
  \item Basic syntax:
    \begin{python}
        if condition:
            statement 1
            statement 2
            ...
        elif condition2:
            statement 3
            ...
        else:
            ...
    \end{python}
  \item Lines below the \obj{if} must be indented the same
  \item In an \obj{if} statement, any of the following are considered
    \obj{False}: \obj{None}, 0, empty sequence or mapping
  \end{itemize}
\end{frame}

\begin{frame}[fragile]{Conditional loop}
  \begin{python}
      while condition:
          statement 1
          statement 2
          ...
  \end{python}
\end{frame}

\begin{frame}[fragile]{Iterative loop}
  \begin{python}
      for var in sequence:
          statement 1
          statement 2
          ...
  \end{python}
\end{frame}

\begin{frame}[fragile]{Integer sequences}
  \begin{itemize}
  \item Special \obj{range} type that represents integer sequences
  \item For example, all integers from 0 to 100 in intervals of 4
    \begin{python}
        for x in range(0, 100, 4):
            print(x)
    \end{python}
  \end{itemize}
\end{frame}

\begin{frame}[fragile]{Iteration functions}
  \obj{zip} creates a list of tuples
  \begin{python}
    numbers = [3, 2, 5, 2]
    words = ['this', 'workshop', 'is', 'cool']
    for num, word in zip(numbers, words):
        print(num, word)
  \end{python}
  \vfill
  \obj{enumerate} gives an index of iteration
  \begin{python}
    for i, word in enumerate(words):
        print('Word {} is {}'.format(i, word))
  \end{python}
\end{frame}

\begin{frame}[fragile]{Comprehensions}
  \begin{python}
    sq = [x**2 for x in range(100)]

    even_sq = [x for x in sq if x % 2 == 0]

    sines = {x: math.sin(x) for x in range(10)}
  \end{python}
\end{frame}

\begin{frame}[fragile]{Exceptions}
  \begin{python}
      try:
          x = int(input("Enter a number: "))
      except ValueError:
          print("Not a valid number!")
  \end{python}
\end{frame}

\begin{frame}[fragile]{Functions}
  \begin{python}
      def add_numbers(x, y=0):
          return x + y

      print(add_numbers(10, 20))
      print(add_numbers(15))
  \end{python}
\end{frame}

\begin{frame}[fragile]{Packing and unpacking}
  \begin{tcblisting}{minted language=ipythonconsole,
      minted options={fontsize=\footnotesize}}
      In [1]: def f(*args, **kwargs):
         ...:     print(args)
         ...:     print(kwargs)
         ...:

      In [2]: f(2017, 'canada', 3.0, president='trump',
                pm='trudeau')
      (2017, 'canada', 3.0)
      {'president': 'trump', 'pm': 'trudeau'}

      In [3]: foo = [10, 20, 30]

      In [4]: f(*foo)
      (10, 20, 30)
      {}
  \end{tcblisting}
\end{frame}

\begin{frame}[fragile]{Classes}
  \begin{python}
      class Dog(object):
          def __init__(self, breed):
              self.breed = breed

          def bark(self):
              print('Woof!')

          def show_breed(self):
              print(self.breed)
  \end{python}
\end{frame}

\begin{frame}[fragile]{Special methods}
  \begin{python}
      class Particles(object):
          def __init__(self, names):
              self.names = names

          def __len__(self):
              return len(self.names)
  \end{python}
\end{frame}

\begin{frame}{Special methods}
  Can implement lots of fancy behavior:
  \begin{table}
    \centering
    \begin{tabular}{ll}
      \toprule
      You write... & Python calls... \\
      \midrule
      \obj{x()} & \obj{x.\_\_call\_\_()} \\
      \obj{str(x)} & \obj{x.\_\_str\_\_()} \\
      \obj{s in x} & \obj{x.\_\_contains\_\_(s)} \\
      \obj{x + y} & \obj{x.\_\_add\_\_(y)} \\
      \obj{x == y} & \obj{x.\_\_eq\_\_(y)} \\
      \obj{copy(x)} & \obj{x.\_\_copy\_\_()} \\
      \bottomrule
    \end{tabular}
  \end{table}
\end{frame}

\begin{frame}{Python Standard Library}
  \begin{itemize}
  \item \obj{re} --- regular expression
  \item \obj{collections} --- container datatypes
  \item \obj{math} --- mathematical functions
  \item \obj{random} --- pseudo-random numbers
  \item \obj{os, glob, shutil} --- working with files
  \item \obj{subprocess} -- subprocess management
  \end{itemize}
\end{frame}

\begin{frame}{Scientific Python Ecosystem}
  Doing science in Python usually requires going beyong the core language and
  standard library. Commonly used third-party packages include:
  \begin{itemize}
  \item NumPy: multi-dimensional arrays, linear algebra, random sampling, FFT
  \item SciPy: integration, optimization, interpolation, statistics, image
    processing
  \item Pandas: data analysis
  \item Matplotlib: MATLAB-like plotting
  \item Jupyter/IPython: interactive shell, notebooks
  \item SymPy: symbolic mathematics
  \item uncertainties: propagation of uncertainty
  \end{itemize}
\end{frame}

\begin{frame}{NumPy arrays}
  \begin{itemize}
  \item OpenMC's Python API uses numpy arrays extensively
  \item \obj{ndarray} class represent N-dimensional array of items of same
    type and size
  \item Array operations are optimized to provide performance closer to a
    low-level language
  \item Fancy indexing
  \item Broadcasting
  \item Lots of syntax inspired by MATLAB
  \end{itemize}
  % Show sin, cos, etc.
  % Show column slicing
  % Show transpose
\end{frame}

\end{document}
