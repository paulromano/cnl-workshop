\documentclass[noamssymb,svgnames]{beamer}
\usecolortheme{beaver}
\useinnertheme[shadow]{rounded}
\usefonttheme{serif}
\usefonttheme{professionalfonts}

\usepackage[bitstream-charter]{mathdesign} % Use BT Charter font
\usepackage[T1]{fontenc}                   % Use T1 encoding instead of OT1
\usepackage[utf8]{inputenc}                % Use UTF8 input encoding
\usepackage{microtype}                     % Improve typography
\usepackage{booktabs}
\usepackage[binary-units]{siunitx}
\usepackage{tikz}
\usetikzlibrary{shapes.geometric}

\usepackage{hyperref}
\hypersetup{pdfstartview=Fit}

% BEAMER CONFIGURATION ---------------------------------------------------------
\setbeamerfont{block title}{size=\normalsize}
\setbeamerfont{block body}{size=\scriptsize}
\setbeamercolor{block title}{fg=darkred,bg=gray!10!white}
\setbeamercolor*{item}{fg=darkred}

% TCOLORBOX / MINTED CONFIGURATION ---------------------------------------------
\usepackage[many,minted]{tcolorbox}
\NewTCBListing{shell}{ O{} }{%
  listing only,minted language=shell,%
  minted options={fontsize=\footnotesize},#1}
\tcbset{listing only}
\setminted{
  mathescape,
  autogobble,
  fontfamily=courier,
  framesep=2mm
}

% ------------------------------------------------------------------------------
\title{Compiling and Installing}

\institute{\includegraphics[width=2in]{../images/openmc_logo.png}}

\date{OpenMC Workshop \\ Canadian Nuclear Laboratories \\ March 14--16, 2017}

% ------------------------------------------------------------------------------
\begin{document}

\frame{\titlepage}

% ------------------------------------------------------------------------------

\begin{frame}[fragile]{Obtaining OpenMC}
  \begin{enumerate}
  \item Download tarball/zip from \url{https://github.com/mit-crpg/openmc}
  \item Clone source using \texttt{git}:
    \begin{shell}
      git clone https://github.com/mit-crpg/openmc
    \end{shell}
  \item Conda forge
    \begin{shell}
        conda config --add channel conda-forge
        conda install openmc
    \end{shell}
  \end{enumerate}
\end{frame}

\begin{frame}[fragile]{Core Prerequisites}
  \begin{table}
    \centering
    \footnotesize{
    \begin{tabular}{lll}
      \toprule
      \textbf{Component} & \textbf{Needed for} \\
      \midrule
      CMake & Platform-independent build tools \\
      Fortran compiler (gfortran) & The core codebase is written in Fortran 2008 \\
      C compiler (gcc) & Faddeeva library is written in C \\
      C++ compiler (g++) & pugixml library is written in C++ \\
      HDF5 & Portable binary I/O \\
      \textcolor{gray}{MPI} & \textcolor{gray}{Distributed-memory parallelism} \\
      \bottomrule
    \end{tabular}
    }
  \end{table}
  \vfill

  On Debian derivatives (Ubuntu), prerequisites can be obtained with:
  \begin{shell}
    sudo apt install cmake gfortran g++ libhdf5-dev
  \end{shell}
  \vfill

  For converting plot images:
  \begin{shell}
    sudo apt install imagemagick
  \end{shell}
\end{frame}

\begin{frame}{Python Prerequisites}
  \begin{table}
    \centering
    \small{
    \begin{tabular}{lcl}
      \toprule
      \textbf{Package} & \textbf{Version} & \textbf{Description} \\
      \midrule
      Python & 2.7 / 3.2+ &  \\
      numpy & 1.9+ & N-dimensional arrays \\
      h5py & 2.5+ & HDF5 file support \\
      scipy &  & Sparse matrices, k-d tree, Faddeeva function \\
      matplotlib &  & Plotting support \\
      pandas & 0.16+ & Tally DataFrames \\
      uncertainties &  & Nuclear data uncertainties \\
      Cython & Resonance reconstruction \\
      \bottomrule
    \end{tabular}
    }
  \end{table}
\end{frame}

\begin{frame}[fragile]{Using conda for Python}
  Download and install miniconda:
  \begin{shell}[minted options={fontsize=\tiny}]
    wget https://repo.continuum.io/miniconda/Miniconda3-latest-Linux-x86_64.sh
    bash Miniconda3-latest-Linux-x86_64.sh
  \end{shell}
  \vfill
  Make sure commands are on your path:
  \begin{shell}
    export PATH="$HOME/miniconda3/bin:$PATH"
  \end{shell}
  \vfill
  Create and activate environment with packages needed:
  \begin{shell}[minted options={fontsize=\tiny}]
    conda create -n openmc numpy scipy h5py matplotlib pandas cython
    source activate openmc
    pip install uncertainties
  \end{shell}
\end{frame}

\begin{frame}[fragile]{Building OpenMC}
  From root directory of repository/tar/zip:
  \begin{shell}
    mkdir build
    cd build
    cmake ..
    make
  \end{shell}
  Now you should have a \texttt{bin} directory with
  \textcolor{green!60!black}{\texttt{openmc}} inside it
\end{frame}

\begin{frame}[fragile]{Build options}
  \begin{table}
    \centering
    \footnotesize{
    \begin{tabular}{ll}
      \toprule
      \textbf{Option} & \textbf{What it does} \\
      \midrule
      debug & Compile with debug flags \\
      optimize & Turn on compiler optimization flags \\
      openmp & Enable shared-memory parallelism with OpenMP \\
      profile & Compile with profiling flags \\
      coverage & Compile with coverage analysis flags \\
      mpif08 & Use Fortran 2008 MPI interface \\
      \bottomrule
    \end{tabular}
    }
  \end{table}
  \vfill
  Options are passed as followed:
  \begin{shell}
    cmake -Ddebug=on ..
  \end{shell}
  \vfill
  If you want to see what's really going on:
  \begin{shell}
    make VERBOSE=1
  \end{shell}
\end{frame}

\begin{frame}[fragile]{Building on macOS}
  \begin{itemize}
  \item Recommend obtaining packages via \texttt{brew}
    \begin{shell}
      brew install cmake gcc
    \end{shell}
  \item HDF5 requires the science ``tap''
    \begin{shell}
      brew tap homebrew/science
      brew install hdf5 --with-fortran2003
    \end{shell}
  \item Commands for building are the same!
  \end{itemize}
\end{frame}

\begin{frame}{Building on Windows}
  Recommended approach:
  \begin{itemize}
  \item Bash on Ubuntu on Windows 10 Anniversary
  \end{itemize}
  Previous attempts with limited success:
  \begin{itemize}
  \item Cygwin
  \item MinGW
  \end{itemize}
\end{frame}

\begin{frame}[fragile]{Installing}
  Once you have successfully built \textcolor{green!60!black}{\texttt{openmc}}:
  \begin{shell}
    make install
  \end{shell}
  This requires administrative privileges. To install locally:
  \begin{shell}
    cmake -DCMAKE_INSTALL_PREFIX=/somewhere/local ..
    make
    make install
  \end{shell}
  Under the hood, \texttt{make install} calls:
  \begin{shell}
    python setup.py install
  \end{shell}
\end{frame}

\begin{frame}[fragile]{Obtaining cross section data}
  OpenMC provides several scripts for obtaining/converting cross section files:
  \begin{table}
    \centering
    \scriptsize{
    \begin{tabular}{ll}
      \toprule
      \textbf{Script} & \textbf{Description} \\
      \midrule
      \texttt{openmc-get-nndc-data} & Download/convert NNDC ENDF/B-VII.1 ACE files \\
      \texttt{openmc-get-jeff-data} & Download/convert JEFF 3.2 ACE files \\
      \texttt{openmc-convert-mcnp70-data} & Convert MCNP ENDF/B-VII.0 data (endf70[a-k]) \\
      \texttt{openmc-convert-mcnp71-data} & Convert MCNP ENDF/B-VII.1 data (endf71x) \\
      \texttt{openmc-ace-to-hdf5} & Convert arbitrary set of ACE files \\
      \bottomrule
    \end{tabular}
    }
  \end{table}
  \vfill
  For convenience, set the following environment variable:
  \begin{shell}[minted options={fontsize=\scriptsize}]
    export OPENMC_CROSS_SECTIONS=<path>/cross_sections.xml
  \end{shell}
\end{frame}

\end{document}
